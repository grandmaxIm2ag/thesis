\nomenclature{$\Sigma$}{Un alphabet}
\nomenclature{$Q$}{Un ensemble d'états}
\nomenclature{$F$}{Un ensemble d'état finaux}
\nomenclature{$\delta$}{Une fonction de transition}
\nomenclature{$q_0$}{Un état initial}
\nomenclature{$q$}{Un état}
\nomenclature{$A$}{Un automate}
\nomenclature{$L$}{Un langage}
\nomenclature{$L(A)$}{Le langage accepté par $A$}
\nomenclature{$\pi$}{Une partition}
\nomenclature{$\lambda$}{Une séquence vide}
\nomenclature{$S_P(L)$}{L'ensemble des préfixes courts}
\nomenclature{$N(L)$}{le noyau d'un langage}
\nomenclature{$I^{+}$}{Un échantillon positif}
\nomenclature{$I^{-}$}{Un échantillon négatif}
\nomenclature{$A(L)$}{L'automate canonique du langage $L$}
\nomenclature{$PTA(I_+)$}{L'arbre accepteur des préfixes}
\nomenclature{$MCA(I_+)$}{L'automate canonique maximal}
\nomenclature{$UA$}{L'automate universel}
\nomenclature{$IS$}{L'état initial d'un problème de planification}
\nomenclature{$G$}{Le but d'un problème de planification}
\nomenclature{$\rho_o$}{La pré-condition d'un opérateur $o$}
\nomenclature{$\epsilon_o$}{Les effets d'un opérateur $o$}
\nomenclature{$\epsilon_o^{+}$}{Les effets positifs d'un opérateur $o$}
\nomenclature{$\epsilon_o^{-}$}{Les effets négatifs d'un opérateur $o$}
\nomenclature{$\Phi_o$}{Les paramètres d'un opérateur $o$}
\nomenclature{$P$}{Un ensemble de prédicats}
\nomenclature{$A^{*}$}{L'automate à apprendre}
\nomenclature{$D$}{Un domaine de planification}
\nomenclature{$D^{*}$}{Le domaine de planification à apprendre}
\nomenclature{$S$}{Une séquence d'actions}
\nomenclature{$B$}{La blackbox}
\nomenclature{$R$}{Un ensemble de contraintes par paires}
\nomenclature{$\mu$}{Un fonction de mapping}